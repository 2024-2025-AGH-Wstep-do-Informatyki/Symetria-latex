
\setlength{\parskip}{10pt}

\section{Wiktoria Gajos}
\label{sec:wiktoriagajos}

\setlength{\parskip}{10pt}
\subsection{Mathematical Expression}

Formula for the Discriminant:
\flushleft
The discriminant, often called "delta," is a distinguishing value of a quadratic trinomial.
Given a quadratic function in the form:

$f(x) = ax^2 + bx + c$

where a, b and c are the coefficients of the quadratic function, and $a \ne 0$,
the formula for the discriminant is as follows:

$\Delta = b^2 - 4ac$

\subsection{Photo of an Idol}
(see Figure~\ref{fig:Taesan}).

\begin{figure}[h]
\centering
    \includegraphics[width=0.5\textwidth]{pictures/zdjęcia_Wiktoria/Taesan.jpg}
    \caption{Taesan from Boynextdoor}
    \label{fig:Taesan}
\end{figure}

\newpage
\subsection{Boynextdoor - members}
(see Figure~\ref{fig:BND}).

\begin{figure}[h]
    \centering
    \includegraphics[width=\linewidth]{pictures/zdjęcia_Wiktoria/BND_all.jpg}
    \caption{All members of Boynextdoor}
    \label{fig:BND}
\end{figure}


\begin{table}[h]
    \centering
    \caption{Members of the Band Boynextdoor }
    \label{tab:band_members}
    \begin{tabular}{|c|c|c|c|}
        \hline
        \textbf{Stage Name} & \textbf{Real Name} & \textbf{Date of Birth} & \textbf{Position in Band} \\
        \hline
         Jaehyun & Myung Jae Hyun & December 4th, 2003 & Leader\\
        \hline
        Sungho & Park Sung Ho & September 4th, 2003 & Main Vocalist\\
        \hline
        Riwoo & Lee Sang Hyuk & October 22nd, 2003 & Main Dancer\\
        \hline
        Taesan & Han Dong Min & August 10th, 2004 & N/A\\
        \hline
        Leehan & Kim Dong Hyun & October 20th, 2004 & N/A\\
        \hline
    \end{tabular}
\end{table}

\newpage
\subsection{What to bring to a Concert}
\subsubsection{Numbered List}

Here is a list of essential items to bring to a concert:

\begin{enumerate}
    \item  Tickets
    \item ID
    \item Comfortable clothes
    \item Comfortable shoes
    \item Water bottle
    \item Portable charger
    \item Earplugs
    \item Credit card
    \item PHONE!!!

\end{enumerate}

\subsubsection{Unordered List}
\begin{itemize}
    \item Rain jacket
    \item Hat or cap
    \item Small backpack
    \item Friends or family :)
\end{itemize}

\subsection{Information about BOYNEXTDOOR}

\onehalfspacing 

\textbf{BOYNEXTDOOR} (see Figure \ref{fig:BND}) (can be shortened as BND) is a South Korean 6-member boy group under KOZ Entertainment who debuted on May 30th, 2023, with the single album, WHO!. The 6 members are Jaehyun, Sungho, Riwoo, Taesan (see Figure  \ref{fig:Taesan}), Leehan, and Woonhak (information about them is in a Table \ref{tab:band_members}). They made their Japanese debut on July 10th, 2024, with the single, AND. The group name explains that they’re the boys who live next door. They will be singing songs that a lot of people can hopefully relate to in their daily lives, as friendly as their name. \par

Official Greeting: \textit{“Who’s there? BOYNEXTDOOR! Hello, we’re BOYNEXTDOOR!“} \par

BOYNEXTDOOR Official Fandom Name: \underline{ONEDOOR}. The fandom name explanation states that the fans of BOYNEXTDOOR are the only ONEDOOR that can connect BOYNEXTDOOR to the world. With ONEDOOR, BOYNEXTDOOR will open the door to a bigger world and move forward towards a shared dream and visions for the future.


